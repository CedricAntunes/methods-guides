\documentclass[10pt]{article}

\usepackage{amsmath}
\usepackage[margin=1in]{geometry}
\usepackage{booktabs}
\usepackage{siunitx}
\usepackage[capitalise]{cleveref}
\usepackage{lipsum}

\begin{document}

As an example consider the two equations
%
\begin{align}
  \langle u, v \rangle & = \langle f, v\rangle\\
                       & = G(v)
\end{align}
%
versus the following
%
\begin{align}
\langle u, v \rangle & =
\langle f, v\rangle\\
& = G(v).
\end{align}

In addition compare the tabular environments~\cref{tab:exptry1,tab:exptry2}.
%
\begin{table}[!ht]
  \centering
  \begin{tabular}{|l||c|c|c|c|}
    \hline
& $n$ &$t$& $\rho$ & $m$\\
    \hline
    \hline
experiment 1 & 19929  & 0.32 & 0.8 & 55\\
    \hline
    experiment 2 &  7729292      & 0.78       & 0.7 & 85\\
    \hline
  exp 3         & 888173928 & 1.25 & 0.65 & 2\\
    \hline
  \end{tabular}
  \caption{3 experiments}\label{tab:exptry1}
\end{table}
%
\begin{table}[!ht]
  \centering
  \begin{tabular}{lrllr}
  \toprule
                 & \multicolumn{1}{c}{$n$}
                 & \multicolumn{1}{c}{$t$}
                 & \multicolumn{1}{c}{$\rho$}
                 & \multicolumn{1}{c}{$m$} \\
  \midrule
    experiment 1 & \num{    19929} & 0.32 & 0.8    & 55  \\
    experiment 2 & \num{  7729292} & 0.78 & 0.7    & 85  \\
    experiment 3 & \num{888173928} & 1.25 & 0.65   & 2   \\
  \bottomrule
  \end{tabular}
  \caption{3 experiments}\label{tab:exptry2}
\end{table}

\lipsum[1-1]
%
\begin{equation}
  A x = b
\end{equation}
%
\lipsum[2-2]
%
\begin{align}
  A x = b
\end{align}
%
\lipsum[3-3]

Consider the PDE with boundary conditions
% TODO should we add extra alignment?
\begin{subequations}
  \begin{align}
    -\nabla \cdot \nabla u & = f \quad \text{in $\Omega$}\\
                         u & = g \quad \text{on $\partial\Omega$}.
  \end{align}
\end{subequations}
Then again we also have the following in 2D\@:
\begin{equation}
\begin{split}
  \mathcal{L} u & = -\Delta u\\
                & = -\nabla \cdot \nabla u\\
                & = -u_{xx} -u_{yy}
\end{split}
\end{equation}
\end{document}
